\section{Introduction}



A brief introduction about the functioning of  a gyrotron in general, as well as an explanation on the MIG mode of operation and the trapping phenomena that can occur in the latter, leading to the formation of electron clouds, are present in \ref{Section_gyrotron}. The Trapped Electrons eXperiment TREX and why it is of importance in the context of the trapping of electron clouds is described in \ref{TREX_Section}. A description of the code used to model the electron trapping through solving the collisional Vlasov-Poisson, FENNECS, is detailed in \ref{fennecs_section}. The main theoretical considerations regarding Ion-Induced Electron emissions IIEE, as well as the choice of a relevant model for them are presented in \ref{IIEE_section}. The details of the numerical implementation of the IIEE as an additional module for the FENNECS code are shown in \ref{Implementation_Section}.  Finally, the module robustness is tested in \ref{CodeTest_section}.  Regarding the formation of electron clouds and the influence of IIEE on the latter is treated in \ref{Cloud_section}, in several geometries, including the GT170 MIG from the ITER gyrotrons. 