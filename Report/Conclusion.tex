\section{Conclusion}
In this report, different geometries have been introduced to study comparatively the electron cloud formation with or without ion-induced electron emissions, in magnetic potential wells. Only one kind of ions was used for the simulations, and the yield was implemented for ions impinging on aluminum only. These simulations were conducted while the TREX experiment is built, to help designing the latter, and to predict some of its results as well.\\

\noindent Results have shown that when the electrons were induced by ions in the trapping region, the cloud densities were  affected, being in some cases, about twice as high as without ion-induced electrons. The density could only be increased in the cloud localised near the cathode, where ions impact. The formation times were shortened under considerations of IIEE, implying that the clouds can form faster, but still on the same time scale. Regarding the observed currents, since the difficulty to maintain the power supply raises with the current in the chamber, the effect of IIEE on the total collected current was investigated. In the end, it turned out that it could be increased by about $20\%$, confirming that the effect of IIEE is somewhat important. However, the order of magnitude of both the densities and the currents was preserved under considerations of IIEE. Thus, being costly to simulate, and their influence being moderate, they could be neglected when considering the formation of clouds in the design of power-supply devices to power gyrotrons. 
