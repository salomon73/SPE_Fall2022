\section{Theory}\label{Theory}

\subsection{Gyrotron guns}
\begin{itemize}
\item{A few words on gyrotrons}
\item{What they are used for}
\item{GT-170: image}
\item{Possible disruptions in the use of gyrotrons}
\item{Has this been quantified ? Time scale ? Densities ? Clouds}
\end{itemize}

\subsection{The FENNECS code}
\begin{itemize}
\item{Particle In Cell code}
\item{Brief description of implementation}
\item{Vlasov Poisson system of equations}
\end{itemize}

\subsection{Ion-Induced Electron-Emissions}
\begin{itemize}
\item{Schou's model}
\item{Kinetic emissions}
\item{Low energies: potential emissions}
\item{Possible cascade phenomena}
\end{itemize}

\subsection{Trapped Electrons EXperiment TREX}
\begin{itemize}
\item{Description of the experiment}
\item{What we hope to see}
\item{Comparisons with our module ?}
\end{itemize}

\subsection{Numerical implementation}
\begin{itemize}
\item{Find tabulated values of energy loss (material dependant)}
\item{Find tabulated values for electronic yield}
\item{Fit these values with energy polynomials of various degrees}
\item{Poisson distribution - random numbers with $\lambda=\gamma(E)$}
\item{Invert Buneman algorithm}
\item{Generate electrons at last position of lost ion with \# of electrons following Poisson}
\end{itemize}



In order to implement numerically Schou's model from Eq.(\ref{}), it was necessary to obtain reference values for the electronic yield, as a function of incident ions' energies. Tabulated values for the energy loss $dE/dx$ of protons in various materials were extracted from \cite{Janni_vol1, Janni_vol2}. To be consistent with the TREX experiment plans (See \ref{TREX_sec}), attention was drawn on 304 stainless steel $^{304}SS$, copper $Cu$ and aluminum $Al$. 
