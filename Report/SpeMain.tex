% Document class 
%=============
\documentclass[aps,prc,floatfix,showkeys,nofootinbib]{revtex4-1}
% Packages 
%========
\usepackage[T1]{fontenc}
\usepackage{amsmath}
\usepackage{amssymb}
\usepackage{mathtools}
\usepackage{booktabs}
\usepackage{siunitx}
\usepackage[referable]{threeparttablex}
\usepackage{longtable}
\usepackage{hyperref}
\usepackage{graphicx}
\usepackage{enumerate}
\usepackage{color}
\usepackage{tikz}
\usepackage{pgfplots}
\usetikzlibrary{arrows.meta}
\usepgfplotslibrary{colormaps}
\usepackage{hhline}
\usepackage{multirow}
\usepackage{braket}
\pgfplotsset{
    compat=newest,
    colormap={mycolormap}{color=(lightgray) color=(white) color=(lightgray)}
}
\def\bibsection{\section*{\refname}} 
% New commands
%=============
\newcommand{\beq}{\begin{equation}}
\newcommand{\eeq}{\end{equation}}
\newcommand{\bes}{\begin{split}}
\newcommand{\ees}{\end{split}}
\newcommand{\BB}{\textbf{B}}
\newcommand{\jj}{\textbf{j}}
\newcommand{\nn}{\textbf{n}}
\newcommand{\rr}{\textbf{r}}
\newcommand{\vv}{\textbf{v}}
\newcommand{\g}{\textbf{g}}
\newcommand{\x}{\times}
\newcommand{\ii}{\iota}
\newcommand{\iotaa}{\bar{\iota}}
\newcommand{\dd}{\partial}
\newcommand{\Div}{\nabla\cdot}
\newcommand{\grad}{\nabla}
\newcommand{\rot}{\nabla \times}
\newcommand{\modB}{\|B\|}
\newcommand{\doubleint}{\int_{0}^{2\pi/N_{fp}}\int_{0}^{2 \pi} d\theta d\phi}
\newcommand{\IInt}{\int \int d \theta d\phi}
\newcommand{\FourierModB}{\sum_{m,n} B_{mn} \cos(m\theta - n  N_{fp}\phi)}
\newcommand{\inscale}{0.495}
\newcommand{\outscale}{0.8}
\newcommand{\myscale}{0.38}
\newcommand{\myscalea}{0.32}
\renewcommand*{\thefootnote}{\alph{footnote}}
% Paths
%=====
\graphicspath{{./IllustrationsTheory/}{./IllustrationsResults/}}  


% Document's body
%==============
%-----------------DOCUMENT------------------------------
%----------------------------------------------------------------
\begin{document}

\title{Numerical study of the influence of ion-induced electrons on the dynamics of electron clouds in gyrotron-like geometries}

\author{S.~Guinchard} 
\email{salomon.guinchard@epfl.ch}
\affiliation{Section de Physique, Ecole Polytechnique Fédérale de Lausanne, Lausanne, Suisse}

\collaboration{Supervised by G.~Le Bars and J.~Loizu}
\altaffiliation{Swiss Plasma Center, Ecole Polytechnique Fédérale de Lausanne, Lausanne, Suisse}

\date{\today}

%------------------ABSTRACT------------------------------
%----------------------------------------------------------------
\begin{abstract}
In this document, the influence of ion-induced electron emissions (IIEE) on the formation of electron clouds in several gyrotron-like geometries is reviewed. Results are compared to the case without taking account for IIEE. The implemented geometries correspond to electrode designs from the TREX experiment \cite{TREX},  currently being built on site at EPFL. The results are obtained by mean of the highly parallelised FENNECS code \cite{fennecs}, in which a module was implemented to deal with these IIEE. Finally, a cloud formation study is conducted in the refurbished geometry from the gyrotron GT-170, part of the ITER plan for ECRH. 
\end{abstract}
\keywords{Gyrotron, IIEE, ion-induced electron emissions, FENNECS, TREX, Magnetron Injection Gun, GT-170, magnetic potential well, trapping, electron trapping}

{
\let\clearpage\relax
\maketitle
\sloppy
}



%----------------------- MAIN --------------------------------
%----------------------------------------------------------------
\section{Introduction}



A brief introduction about the functioning of  a gyrotron in general, as well as an explanation on the MIG mode of operation and the trapping phenomena that can occur in the latter, leading to the formation of electron clouds, are present in \ref{Section_gyrotron}. The Trapped Electrons eXperiment TREX and why it is of importance in the context of the trapping of electron clouds is described in \ref{TREX_Section}. A description of the code used to model the electron trapping through solving the collisional Vlasov-Poisson, FENNECS, is detailed in \ref{fennecs_section}. The main theoretical considerations regarding Ion-Induced Electron emissions IIEE, as well as the choice of a relevant model for them are presented in \ref{IIEE_section}. The details of the numerical implementation of the IIEE as an additional module for the FENNECS code are shown in \ref{Implementation_Section}.  Finally, the module robustness is tested in \ref{CodeTest_section}.  Regarding the formation of electron clouds and the influence of IIEE on the latter is treated in \ref{Cloud_section}, in several geometries, including the GT170 MIG from the ITER gyrotrons. 
\newpage

\section{Theory}\label{Theory}

\subsection{Gyrotron guns}



\section{Results}\label{results}

\subsection{Code testing}

The IIEE module being implemented, some tests were required in order to check that the code behaves accordingly. To do so, several tests were realised. The first test consisted of statistical results for the yield. To derive statistical results, the simple geometry from \ref{Theory_axy} was implemented, as well as several horizontal slices of ions. These ions were initialised at different radial distances from the cathode. Making use of Eq.(\ref{}), their energy as they hit the electrode is known, and so is the average yield per incident ion. Recall that the yield is a function of the incident ion energy only, through the energy loss in the material. 

\beq
\gamma(E) = \Lambda_{exp}\cdot \frac{dE}{dx}\Bigg|_i
\eeq 

\noindent For all three materials, the three slices of ions were inialised along the axial direction, thus enabling to show that the electron generation from the code, over the loss of ions is independent of the axial position, as expected. Fig.(\ref{flat_slice}) shows the results for $H_2^{+}$ cations impinging on the cathode made either of $^{304}$SS, Cu or Al. The potential bias applied between the two electrodes was $\Delta \phi=20$ kV. The magnetic field is uniform, with field lines parallel to the electrodes, and $B=0.21$ T, as it is a value of the order of magnitude as used in gyrotrons. The radial positions of the electrode were $r_a=0.001$ m for the cathode and $0.01$ m for the anode, and the ions were generated at $r_1 = 0.003$ m, $r_2=0.005$ m and $r_3=0.008$ m. Comparison of the theoretical average electron yield and the one obtained from the IIEE module is exposed in Table.(\ref{tab_stat}). 

\begin{figure}[h!]
\centering
	\includegraphics[width = 1.0 \textwidth]{Flatslice_stats.eps}
	\caption{\label{flat_slice} }
\end{figure}  

\begin{table}[h]
  \centering
  \renewcommand{\arraystretch}{1.2}
  \begin{tabular}{|p{5cm}|c|c|c|c|}
    \hline
    \multirow{2}{5cm}{\textbf{Mode Transition Times Duration (Typical)}} & \multicolumn{3}{c|}{\textbf{Material}} \\
    % \hline
    % \textbf{Inactive Modes} & \textbf{Description}\\
    \cline{2-5}
    & \textbf{$^{304}$SS} & \textbf{Cu} & \textbf{Al} & \textbf{[MHz]}\\
    %\hhline{~--}
    \hline
    $E$ & 1365 & 1260 & 1024 &  \\ \hline
    $\gamma_{th}$ & TBD & TBD & TBD &  \\ \hline
    $\gamma_{iiee}$ & TBD & TBD & TBD & \\ \hline
    $\epsilon_{rel}$ & TBD & TBD & TBD & \\ \hline
  \end{tabular}
\end{table}


\subsection{Cloud formation}

\subsection{Further implementation}

\section{Conclusion}
In this report, different geometries have been introduced to study comparatively the electron cloud formation with or without ion-induced electron emissions, in magnetic potential wells. Only one kind of ions was used for the simulations, and the yield was implemented for ions impinging on aluminum only. These simulations were conducted while the TREX experiment is built, to help designing the latter, and to predict some of its results as well.\\

\noindent Results have shown that when the electrons were induced by ions in the trapping region, the cloud densities were  affected, being in some cases, about twice as high as without ion-induced electrons. The density could only be increased in the cloud localised near the cathode, where ions impact. The formation times were shortened under considerations of IIEE, implying that the clouds can form faster, but still on the same time scale. Regarding the observed currents, since the difficulty to maintain the power supply raises with the current in the chamber, the effect of IIEE on the total collected current was investigated. In the end, it turned out that it could be increased by about $20\%$, confirming that the effect of IIEE is somewhat important. However, the order of magnitude of both the densities and the currents was preserved under considerations of IIEE. Thus, being costly to simulate, and their influence being moderate, they could be neglected when considering the formation of clouds in the design of power-supply devices to power gyrotrons. 


\input{Appendix}


\newpage
\bibliographystyle{alpha}
\bibliography{references}
\end{document}
% End of document
%==============