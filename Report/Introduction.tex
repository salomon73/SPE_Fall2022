\section{Introduction}

Over the years, the energy needs have not stopped increasing at a global scale, and the diminishing fossil fuels ressources, as well as ecological concerns, have made emerge new sources for a sustainable energy production. Nuclear fusion is among them. The research in nuclear fusion is a tremendous task that has kept physicists busy for years now. For a nuclear fusion power plant to operate, various complex systems have to  work together, as the plasma need to be produced, heated and confined. Plasma heating is an important field of research, since an optimized heating system is among the keys to reduce the energy consumption and raise the fusion gain factor $Q$. \\

In the past decades, gyrotrons have proven to be essential devices to reach required conditions for nuclear fusion, as they are used for plasma heating and current drive applications, in a long-pulse or continuous operating mode \cite{Fisch}. They are complex devices made of an ensemble of subsystems, each of them being essential for the functioning, and making intervene their own physical phenomena \cite{ITER_gyrotrons}. It is hence essential for each part to function properly, for the whole heating system to be efficient. One particular part of the gyrotron is of interest, since perturbative phenomena can occur as it is used: the so called MIG, or Magnetron Injection Gun. This system is used to produce the electron beam that will later be used to generate millimeter electromagnetic waves. These waves will resonate with the plasma electrons to heat the latter, or induce a toroidal current, that, in Tokamak, will help generating the poloidal magnetic field indispensable for confinement. It has been shown that various phenomena can disrupt the proper operating mode of the MIG, among which the formation of trapped electron clouds \cite{lebars_et_al}. These electron clouds can considerably affect the local electric field, and be responsible for currents that make the power-supply sustaining impossible over sufficient times. It is then in our best interest to understand the formation of these clouds, how it depends on the MIG geometry, to be able to prevent their formation, or neutralise them. With this report, we hope to bring additional information on the knowledge about the formation of electron clouds, as a particular phenomenon, the ion-induced electron emissions, is newly considered. \\

A brief introduction about the functioning of a gyrotron in general, as well as an explanation on the MIG mode of operation and the trapping phenomena that can occur in the latter, leading to the formation of electron clouds, are present in \ref{Section_gyrotron}. The Trapped Electrons eXperiment TREX and why it is of importance in the context of the trapping of electron clouds is described in \ref{TREX_Section}. A description of the code used to model the electron trapping through solving the collisional Vlasov-Poisson, FENNECS, is detailed in \ref{fennecs_section}. The main theoretical considerations regarding Ion-Induced Electron emissions IIEE, as well as the choice of a relevant model for them, are presented in \ref{IIEE_section}. The details of the numerical implementation of the IIEE as an additional module for the FENNECS code are shown in \ref{Implementation_Section}.  Finally, the module robustness is tested in \ref{CodeTest_section}.  Regarding the formation of electron clouds and the influence of IIEE on the latter is treated in \ref{Cloud_section}, in several geometries, including the GT170 MIG from the ITER gyrotrons. 